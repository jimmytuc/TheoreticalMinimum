\documentclass[10pt, landscape]{article}

\usepackage{hyperref}
\usepackage{multicol}
\usepackage[landscape]{geometry}
\usepackage{tabularx}

\begin{document}

	\begin{center}
		\Huge{CLEP Natural Science Cheat Sheet} \\ 
		\Large{By Alexander Ahmann (\href{mailto:alexander.ahmann@outlook.com}{$<$alexander.ahmann@outlook.com$>$}; \href{https://twitter.com/mathmare\_}{@mathmare\_})}
	\end{center}
	
	This is a cheat sheet that I developed to help myself and hopefully others pass the CLEP Natural Sciences exam. This is meant to be used with the \textit{REA: CLEP Natural Sciences Exam (Third Edition; ISBN-13: 978-0-7386-1207-2)} exam guide.

	\begin{multicols}{2}
		\section{Passing the CLEP Natural Sciences Exam (Ch. 1)}
		\section{Evolution and Classification (Ch. 2)}
			\begin{itemize}
				%TODO make a figure for the Miller-Urey Experiment on pg. 18
				\item \textbf{Homologous} structures are physiological structures that exists between two animals because they share a common ancestry. An example of this would be the forelimbs of a salamander and opossum, they are similar because of a common ancestry.
				\item \textbf{Analogous} structures are physiological structures that have a similar function and are shared between two organisms, but they do not share a common ancestry.
				\begin{itemize}
					\item Analogous structures are the product of convergent evolution.
				\end{itemize}
				\item \textbf{Convergent Evolution}: TODO Insert Definition %TODO Insert Definition
				\item \textbf{\textit{Homo sapien sapien}} refers to the modern human.
				\begin{itemize}
					\item Homo sapiens are thought to have evolved from Africa ~100,000 years ago.
					\item \textit{Australopithecus Afarenis} (or the "Lucy" skeleton) refers to the first known skeleton belonging to a branch of bipedal primates that gave rise to "the first true hominids" about 4.5 million years ago.
					\item \textit{Homo Erectus} is considered to be the oldest known fossil of the human genus; it is considered to be about 1.8 million years old. 
				\end{itemize}
				%TODO make a figure for the Geologic Time Scale on pg. 23
				\item \textbf{Gene Pool} refers to the entire collection of genes within a given population.
				\item \textbf{Differential Reproduction} is one mechanism of evolution; this is what occurs when some individuals within a population are more suited for survival, given their environmental conditions. 
				\item \textbf{Mutation} is another mechanism of evolution; this is a change of the DNA sequence of a gene, which results in a change of a particular trait.
				\item \textbf{Genetic Drift} is the third mechanism of evolution; it refers to a change of allele frequency due to chance fluctuations. In a finite population, the gene pool might not reflect the entire number of genetic possibilities of the larger gene pool of the species.
				\item \textbf{Hardy-Weinberg Theorem} states that given a population with one allele $p$ and another allele $q$, the sum of the possible allele frequencies is 1: $$ p + q = 1 $$ which then follows that the frequency of the alleles of a population can be represented by: $$ p^2 + 2pq + q^2 = 1 $$ Here, the dominant allele is $ p^2 $, the heterozygote is $ 2pq $ and the recessive allele is $ q^2 $, they should all add to 1.
				\item \textbf{Allpatric speciation} occurs when two populations are separated geographically.
				\item \textbf{Sympatric Speciation} occurs when two genetically different members of the same species produce a third new species.
				\item \textbf{Punctuated Equilibrium} is a scientific model that proposes that adaptations of species arise suddenly and rapidly. It states that species undergo a long period of equilibrium, which at some point is upset by environmental forces causing a short period of quick mutation and change.
				\item \textbf{Taxonomy} seeks to organize living things into groups based on morphology and genetics (genetics came more recently)
				\item \textbf{Binomial Nomenclature} is what biologists refer to when naming species with two Latin names.
				%TODO Add a graphic of organism classification system on pg. 28
				\item \textbf{Linnaean Taxonomy} is a classification scheme for living organisms, and it uses the following hierarchical system:
				\begin{itemize}
					\item \textbf{Kingdom:} this is the most general category, it contains four kingdoms: \textit{Kingdom Protista}, \textit{Kingdom Fungi}, \textit{Kingdom Animalia}, \textit{Kingdom Plantae}. See table 2-1 for more about the four kingdoms.
					\item \textbf{Phylum:} 
					\item \textbf{Class:}
					\item \textbf{Order:}
					\item \textbf{Family:}
					\item \textbf{Genus:}
					\item \textbf{Species:} this is the most limited category
				\end{itemize}
				\item The acronym \textbf{K}evin, \textbf{P}lease \textbf{C}ome \textbf{O}ver \textbf{F}or \textbf{G}ay \textbf{S}ex can be used to memorise Linnaean ranks for any upcoming exams.\footnote{I stole the "Kevin, Please Come Over For Gay Sex" acronym from \href{https://twitter.com/KingCrocoduck}{@KingCrocoduck} ;-) \\ \href{https://twitter.com/KingCrocoduck/status/883562581901717504}{https://twitter.com/KingCrocoduck/status/883562581901717504}}
				\item If you were to focus on \textit{Kingdom Animalia}, then you will come across nine major phyla; they are:
				\begin{itemize}
					\item \textbf{Porifera:} the sponges
					\item \textbf{Cnidaria:} jellyfish, sea anemones, hydra, etc/
					\item \textbf{Platyhelminthes:} flat worms
					\item \textbf{Nematoda:} round worms
					\item \textbf{Mollusca:} snails, clams, squid, etc.
					\item \textbf{Annelida:} segmented worms (earthworms, leeches, etc.
					\item \textbf{Arthropoda:} crabs, spiders, lobster, millipedes, insects
					\item \textbf{Echinodermata:} sea stars, sand dollars, etc.
					\item \textbf{Chordata:} fish, amphibians, reptiles, birds, mammals, lampreys
				\end{itemize}
				\item \textbf{Urochordata} are animals with a tail cord
				\item \textbf{Cephalochordata} are animals with a head cord
				\item Vertebrata is divided into two super-classes:
				\begin{itemize}
					\item \textbf{Aganatha} which are animals with no jaws
					\item \textbf{Gnathostomata} which are animals with jaws
				\end{itemize}
				\item The Gnathostomata includes these six major classes with the following characteristics:
				\begin{itemize}
					\item \textbf{Chondrichthyes:} fish with a cartilaginous endoskeleton, two-chambered heart, 5-7 gill pairs, no swim bladder or lung, and internal fertilization--such as with sharks, rays, et cetera. 
					\item \textbf{Osteichthyes:} fish with a bony skeleton, numerous vertebrae, swim bladder (usually), two-chambered heart, gills with bony gill arches, and external fertilization (herring, carp, tuna).
					\item \textbf{Amphibia:} animals with a bony skeleton, usually with four limbs having webbed feet with four toes, cold-blooded (ectothermic), large mouth with small teeth, three-chambered heart, separate sexes, internal or external fertilization, amniotic egg (like salamander, frogs, et cetera).
					\item \textbf{Reptilia:} horny epidermal scales, usually have paired limbs with five toes (except limbless snakes), bony skeleton, lungs, no gills, most have a three-chambered heart, separate sexes, mostly egg-laying (oviparous), eggs contain extraembryonic membranes (like snakes, lizards, alligators).
					\item \textbf{Aves:} spindle-shaped body (with head neck, trunk, and tail), long neck, paired limbs, most have wings for flying, four-toed foot, feathers, leg scales, bony skeleton, bones with air cavities, beak, no teeth, four-chambered heart, warm blooded (endothermic), lungs with thin air sacs, separate sexes, egg-laying, eggs have hard calcified shell (birds, ducks, sparrows, etc.)
					\item \textbf{Mammalia:} body covered with hair, glands (sweat, scent, sebaceous, mammary), teeth, fleshy external ears, usually four limbs, four-chambered heart, lungs, larynx, highly developed brain, warm-blooded, internal fertilization, live birth (except for the egg-laying monotremes), milk-producing (cows, humans, platypus, apes, et cetera).
				\end{itemize}
			\end{itemize}
		\section{Cellular and Molecular Biology (Ch. 3)}
		\section{Structure and Function of Plants and Animals; Genetics (Ch. 4)}
		\section{Ecology and Population Biology (Ch. 5)}
		\section{Atomic Chemistry (Ch. 6)}
		\section{Chemistry of Reactions (Ch. 7)}
		\section{Physics (Ch. 8)}
		\section{Energy (Ch. 9)}
		\section{The Universe (Ch. 10)}
		\section{Earth (Ch. 11)}
	\end{multicols}
	
	\section{Appendix: Tables}
	\subsection{Four Kingdoms of the Eukaryota}
		\begin{tabularx}{\textwidth}{| X | X | X | X | X |}
			\hline
			\textbf{Kingdom} & \textbf{No. of Known Phyla/Species} & \textbf{Nutrition} & \textbf{Structure} & \textbf{Included Organisms} \\ \hline
			\textit{Protista} & 27/250,000+ & photosynthesis, some ingestion and absorption & large eukaryotic cells & algae \& protozoa \\ \hline
			\textit{Fungi} & 5/100,000+ & absorption & multicellular (eukaryotic) filaments & mold, mushrooms, yeast, smuts, mildew \\ \hline
			\textit{Animalia} & 33/1,000,000+ & ingestion & multicellular, specialized eukaryotic motile cells & various worms, sponges, fish, insects, reptiles, amphibians, birds, and mammals \\ \hline
			\textit{Plantae} & 10/250,000+ & photosynthesis & multicellular, specialized eukaroytic nonmotile cells & ferns, mosses, woody and nonwoody flowering plants \\ \hline
		\end{tabularx}
		
	\section{Acknowledgments}
	This cheat sheet is heavily based off of the study guide \textit{CLEP Natural Sciences: Earn College Credit with CLEP} (ISBN-13: 978-0-7386-1207-2). No copyright infringement is intended.
	
\end{document}